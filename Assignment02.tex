% Options for packages loaded elsewhere
\PassOptionsToPackage{unicode}{hyperref}
\PassOptionsToPackage{hyphens}{url}
%
\documentclass[
]{article}
\usepackage{amsmath,amssymb}
\usepackage{iftex}
\ifPDFTeX
  \usepackage[T1]{fontenc}
  \usepackage[utf8]{inputenc}
  \usepackage{textcomp} % provide euro and other symbols
\else % if luatex or xetex
  \usepackage{unicode-math} % this also loads fontspec
  \defaultfontfeatures{Scale=MatchLowercase}
  \defaultfontfeatures[\rmfamily]{Ligatures=TeX,Scale=1}
\fi
\usepackage{lmodern}
\ifPDFTeX\else
  % xetex/luatex font selection
\fi
% Use upquote if available, for straight quotes in verbatim environments
\IfFileExists{upquote.sty}{\usepackage{upquote}}{}
\IfFileExists{microtype.sty}{% use microtype if available
  \usepackage[]{microtype}
  \UseMicrotypeSet[protrusion]{basicmath} % disable protrusion for tt fonts
}{}
\makeatletter
\@ifundefined{KOMAClassName}{% if non-KOMA class
  \IfFileExists{parskip.sty}{%
    \usepackage{parskip}
  }{% else
    \setlength{\parindent}{0pt}
    \setlength{\parskip}{6pt plus 2pt minus 1pt}}
}{% if KOMA class
  \KOMAoptions{parskip=half}}
\makeatother
\usepackage{xcolor}
\usepackage[margin=1in]{geometry}
\usepackage{color}
\usepackage{fancyvrb}
\newcommand{\VerbBar}{|}
\newcommand{\VERB}{\Verb[commandchars=\\\{\}]}
\DefineVerbatimEnvironment{Highlighting}{Verbatim}{commandchars=\\\{\}}
% Add ',fontsize=\small' for more characters per line
\usepackage{framed}
\definecolor{shadecolor}{RGB}{248,248,248}
\newenvironment{Shaded}{\begin{snugshade}}{\end{snugshade}}
\newcommand{\AlertTok}[1]{\textcolor[rgb]{0.94,0.16,0.16}{#1}}
\newcommand{\AnnotationTok}[1]{\textcolor[rgb]{0.56,0.35,0.01}{\textbf{\textit{#1}}}}
\newcommand{\AttributeTok}[1]{\textcolor[rgb]{0.13,0.29,0.53}{#1}}
\newcommand{\BaseNTok}[1]{\textcolor[rgb]{0.00,0.00,0.81}{#1}}
\newcommand{\BuiltInTok}[1]{#1}
\newcommand{\CharTok}[1]{\textcolor[rgb]{0.31,0.60,0.02}{#1}}
\newcommand{\CommentTok}[1]{\textcolor[rgb]{0.56,0.35,0.01}{\textit{#1}}}
\newcommand{\CommentVarTok}[1]{\textcolor[rgb]{0.56,0.35,0.01}{\textbf{\textit{#1}}}}
\newcommand{\ConstantTok}[1]{\textcolor[rgb]{0.56,0.35,0.01}{#1}}
\newcommand{\ControlFlowTok}[1]{\textcolor[rgb]{0.13,0.29,0.53}{\textbf{#1}}}
\newcommand{\DataTypeTok}[1]{\textcolor[rgb]{0.13,0.29,0.53}{#1}}
\newcommand{\DecValTok}[1]{\textcolor[rgb]{0.00,0.00,0.81}{#1}}
\newcommand{\DocumentationTok}[1]{\textcolor[rgb]{0.56,0.35,0.01}{\textbf{\textit{#1}}}}
\newcommand{\ErrorTok}[1]{\textcolor[rgb]{0.64,0.00,0.00}{\textbf{#1}}}
\newcommand{\ExtensionTok}[1]{#1}
\newcommand{\FloatTok}[1]{\textcolor[rgb]{0.00,0.00,0.81}{#1}}
\newcommand{\FunctionTok}[1]{\textcolor[rgb]{0.13,0.29,0.53}{\textbf{#1}}}
\newcommand{\ImportTok}[1]{#1}
\newcommand{\InformationTok}[1]{\textcolor[rgb]{0.56,0.35,0.01}{\textbf{\textit{#1}}}}
\newcommand{\KeywordTok}[1]{\textcolor[rgb]{0.13,0.29,0.53}{\textbf{#1}}}
\newcommand{\NormalTok}[1]{#1}
\newcommand{\OperatorTok}[1]{\textcolor[rgb]{0.81,0.36,0.00}{\textbf{#1}}}
\newcommand{\OtherTok}[1]{\textcolor[rgb]{0.56,0.35,0.01}{#1}}
\newcommand{\PreprocessorTok}[1]{\textcolor[rgb]{0.56,0.35,0.01}{\textit{#1}}}
\newcommand{\RegionMarkerTok}[1]{#1}
\newcommand{\SpecialCharTok}[1]{\textcolor[rgb]{0.81,0.36,0.00}{\textbf{#1}}}
\newcommand{\SpecialStringTok}[1]{\textcolor[rgb]{0.31,0.60,0.02}{#1}}
\newcommand{\StringTok}[1]{\textcolor[rgb]{0.31,0.60,0.02}{#1}}
\newcommand{\VariableTok}[1]{\textcolor[rgb]{0.00,0.00,0.00}{#1}}
\newcommand{\VerbatimStringTok}[1]{\textcolor[rgb]{0.31,0.60,0.02}{#1}}
\newcommand{\WarningTok}[1]{\textcolor[rgb]{0.56,0.35,0.01}{\textbf{\textit{#1}}}}
\usepackage{graphicx}
\makeatletter
\def\maxwidth{\ifdim\Gin@nat@width>\linewidth\linewidth\else\Gin@nat@width\fi}
\def\maxheight{\ifdim\Gin@nat@height>\textheight\textheight\else\Gin@nat@height\fi}
\makeatother
% Scale images if necessary, so that they will not overflow the page
% margins by default, and it is still possible to overwrite the defaults
% using explicit options in \includegraphics[width, height, ...]{}
\setkeys{Gin}{width=\maxwidth,height=\maxheight,keepaspectratio}
% Set default figure placement to htbp
\makeatletter
\def\fps@figure{htbp}
\makeatother
\setlength{\emergencystretch}{3em} % prevent overfull lines
\providecommand{\tightlist}{%
  \setlength{\itemsep}{0pt}\setlength{\parskip}{0pt}}
\setcounter{secnumdepth}{-\maxdimen} % remove section numbering
\ifLuaTeX
  \usepackage{selnolig}  % disable illegal ligatures
\fi
\IfFileExists{bookmark.sty}{\usepackage{bookmark}}{\usepackage{hyperref}}
\IfFileExists{xurl.sty}{\usepackage{xurl}}{} % add URL line breaks if available
\urlstyle{same}
\hypersetup{
  pdftitle={Assignment02},
  pdfauthor={Ken Suzuki},
  hidelinks,
  pdfcreator={LaTeX via pandoc}}

\title{Assignment02}
\author{Ken Suzuki}
\date{2023-11-21}

\begin{document}
\maketitle

For this assignment, you will be using the sat.act dataset available
within the psych package in R.

\begin{Shaded}
\begin{Highlighting}[]
\FunctionTok{library}\NormalTok{(psych)}
\NormalTok{d }\OtherTok{\textless{}{-}}\NormalTok{ sat.act }
\FunctionTok{head}\NormalTok{(d)}
\end{Highlighting}
\end{Shaded}

\begin{verbatim}
##       gender education age ACT SATV SATQ
## 29442      2         3  19  24  500  500
## 29457      2         3  23  35  600  500
## 29498      2         3  20  21  480  470
## 29503      1         4  27  26  550  520
## 29504      1         2  33  31  600  550
## 29518      1         5  26  28  640  640
\end{verbatim}

\begin{Shaded}
\begin{Highlighting}[]
\FunctionTok{str}\NormalTok{(d)}
\end{Highlighting}
\end{Shaded}

\begin{verbatim}
## 'data.frame':    700 obs. of  6 variables:
##  $ gender   : int  2 2 2 1 1 1 2 1 2 2 ...
##  $ education: int  3 3 3 4 2 5 5 3 4 5 ...
##  $ age      : int  19 23 20 27 33 26 30 19 23 40 ...
##  $ ACT      : int  24 35 21 26 31 28 36 22 22 35 ...
##  $ SATV     : int  500 600 480 550 600 640 610 520 400 730 ...
##  $ SATQ     : int  500 500 470 520 550 640 500 560 600 800 ...
\end{verbatim}

\hypertarget{variables}{%
\subsubsection{Variables}\label{variables}}

\begin{verbatim}
1) Gender: 1 = males; 2 = females
2) Education: 0 = less than high school; 1 = high school; 2 = post-secondary diploma program; 3 = some university/college; 4 = completed university/college; 5 = graduate work.
3) Age = age in years
4) ACT = American College Test (ranges from 0 to 36)
5) SATQ = Standard Aptitude Test – Quantitative (ranges from 200-800)
6) SATV = Standard Aptitude Test – Verbal (ranges from 200-800)
\end{verbatim}

For the questions below, use α = .10 for any null hypothesis
significance tests.

\hypertarget{part-a-12-marks}{%
\subsubsection{Part A (12 marks)}\label{part-a-12-marks}}

Explore how the interaction between gender and age affects scores on the
SATQ. More specifically,summarize the findings related to this effect,
including the following:

\begin{enumerate}
\def\labelenumi{\arabic{enumi})}
\tightlist
\item
  Set-up the variables/model and print the results (3 marks)
\end{enumerate}

\begin{Shaded}
\begin{Highlighting}[]
\NormalTok{model01 }\OtherTok{\textless{}{-}} \FunctionTok{lm}\NormalTok{(SATQ }\SpecialCharTok{\textasciitilde{}}\NormalTok{ gender }\SpecialCharTok{*}\NormalTok{ age, }\AttributeTok{data =}\NormalTok{ d)}
\FunctionTok{confint}\NormalTok{(model01, }\AttributeTok{level =} \FloatTok{0.90}\NormalTok{)}
\end{Highlighting}
\end{Shaded}

\begin{verbatim}
##                     5 %          95 %
## (Intercept) 549.4006790 695.346999457
## gender      -42.7283407  42.992317966
## age          -0.5654472   4.719193133
## gender:age   -3.1205903  -0.001636873
\end{verbatim}

\begin{Shaded}
\begin{Highlighting}[]
\FunctionTok{summary}\NormalTok{(model01)}
\end{Highlighting}
\end{Shaded}

\begin{verbatim}
## 
## Call:
## lm(formula = SATQ ~ gender * age, data = d)
## 
## Residuals:
##     Min      1Q  Median      3Q     Max 
## -398.59  -81.31   14.99   86.66  232.17 
## 
## Coefficients:
##             Estimate Std. Error t value Pr(>|t|)    
## (Intercept) 622.3738    44.3044  14.048   <2e-16 ***
## gender        0.1320    26.0219   0.005   0.9960    
## age           2.0769     1.6042   1.295   0.1959    
## gender:age   -1.5611     0.9468  -1.649   0.0996 .  
## ---
## Signif. codes:  0 '***' 0.001 '**' 0.01 '*' 0.05 '.' 0.1 ' ' 1
## 
## Residual standard error: 114 on 683 degrees of freedom
##   (13 observations deleted due to missingness)
## Multiple R-squared:  0.03259,    Adjusted R-squared:  0.02834 
## F-statistic: 7.669 on 3 and 683 DF,  p-value: 4.791e-05
\end{verbatim}

\begin{enumerate}
\def\labelenumi{\arabic{enumi})}
\setcounter{enumi}{1}
\tightlist
\item
  Summarize the traditional null hypothesis significance testing results
  (1 mark)
\item
  Plot the interaction, with a summary of visualization (2 marks)
\item
  Conduct a negligible effect test on the interaction (using a
  negligible effect interval of b = -2 to b = 2), and summarize the
  results (3 marks)
\end{enumerate}

\begin{verbatim}
Note that specifying the formula with the following formats will not work in neg.reg: y ~x1*x2 or y ~ x1 + x2 + x1:x2. There are two ways around this: a) compute the interaction term before running the model (e.g., int = x1*x2, using a numeric version of gender), and then use neg.reg (formula = y ~ x1 + x2 + int), or b) compute the confidence intervalfor the interaction separately using confint(model).
\end{verbatim}

\begin{enumerate}
\def\labelenumi{\arabic{enumi})}
\setcounter{enumi}{4}
\tightlist
\item
  Provide an interpretation of the magnitude of the interaction using
  both standardized and unstandardized units, as well as an
  interpretation of magnitude using the proportional distance (specific
  to the negligible effect test). (3 marks)
\end{enumerate}

\end{document}
